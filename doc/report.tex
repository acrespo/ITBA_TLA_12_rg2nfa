\documentclass[11pt,a4paper,titlepage]{article}
\usepackage[utf8]{inputenc}
\usepackage[spanish]{babel}
\usepackage{amsmath}
\usepackage{amsfonts}
\usepackage{amssymb}
\usepackage{makeidx}
\usepackage{graphicx}
\usepackage{tipa}

\title{Conversor de Gramáticas - Autómatas Finitos}
\author{Civile, Juan \and Sneidermanis, Dario \and Alvaro, Crespo}
\date{27 de Mayo del 2012}

\begin{document}

\newcommand{\awesome}[1]{\texttt{\large #1}}

\maketitle
\tableofcontents
\clearpage

\section{Consideraciones}

    Se hizo una pequeña modificación a la sintaxis de los archivos de
    gramáticas, para poder implementarla (los nombres de gramáticas no
    admiten caracteres `\texttt{=}`).

    La sintaxis de los archivos \texttt{dot} fue implementada de una forma más
    relajada, dado que no se implementó la sintaxis completa.

    Se asumió que los simbolos terminales de las gramáticas eran letras
    minúsculas, y los simbolos no terminales letras mayúsculas. Además, debe
    haber al menos una letra mayúscula no usada, para en el estado
    final añadido al transformar la gramática en autómata.

    En los archivos \texttt{dot} se asumió que el primer nodo declarado es el
    inicial, a falta de otros indicadores.

    Finalmente, dado que era necesario compilar los archivos \texttt{dot} para
    generar los gráficos de los autómatas, la correcta ejecución del programa
    solo se asegura en un sistema operativo Linux con la utilidad dot instalada
    y en el path.

\section{Desarrollo}

    Inicialmente se detectó que la parte más compleja del trabajo práctico
    eran las transformaciones a la gramática regular. El paso de autómata
    a gramática regular es bastante sencillo y por lo tanto se dejó para el
    final.

    Esto fue cierto hasta tal punto que ni siquiera se necesitó una estructura
    para representar los autómatas. Estos se transforman en gramáticas
    directamente al ser leidos.

    Se tomo una especial atención a los mensajes de error durante el parseo ya
    que esto servía tanto para el usuario como para nosotros, los
    programadores, para debuggear.

\section{Dificultades}

    Una de las grandes dificultades encontradas fue la decisión de qué
    estructura de datos usar para mantener las gramáticas regulares,
    manteniendo eficiencia, y con la máxima facilidad para efectuar las
    transformaciones necesarias.

    La parte problemática era, más que nada, cómo guardar las producciones,
    manteniendo simpleza tanto en la consulta como en la modificación. En un
    principio se usó un vector de listas enlazadas, pero se termino cambiando
    a un vector de vectores dinámicos, para facilitar el manejo de memoria. Y
    se hizo que varias de las transformaciones trataran a la estructura de
    datos como inmutable.

    Dado que era un proyecto que hacía uso intensivo de diversas estructuras
    de datos, C no era la herramienta perfecta, debido a su falta de librería
    estandar de estructuras de datos. A pesar de que es una interesante tarea
    programar las distintas estructuras clásicas, no eran (creemos) el objetivo
    del trabajo práctico, y freno considerablemente el desarrollo de este.


\section{Extensiones}

    Se podrían añadir conversiones desde y hacie expresiones regulares sin
    mucho trabajo extra. Otra meta más desafiente podría ser transformaciones
    de gramáticas libres de contexto a autómatas de pila.

    En varios lugares se podría mejorar el orden de los algoritmos usando
    árboles balanceados, aunque dado que la cantidad de simbolos esta
    fuertemente restringida por la cantidad de símbolos, no haría un gran
    cambio al tiempo de ejecución.

    Una ultima extensión, es soportar símbolos multicaracteres, para poder
    representar gramáticas/autómatas más extensos.

\end{document}

